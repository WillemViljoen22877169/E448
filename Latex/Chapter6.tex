\chapter[Conclusion]{Chapter 6: Conclusion}\label{chap:last}%Two pages long, what did you see, what do you imagine?
%*******************************************%
\section{Retrospective Overview}\label{sec:retro_ovv}
%*******************************************%
The crux of a modular BMS is its three core components: the Main Controller's power rating, linking the battery pack with external power lines, the scalable monitoring offered by Cell Modules, and the Module Communication System. Together, they form the heartbeat of the BMS, each playing a pivotal role in the system's overall performance and adaptability. The Main Controller has proven pivotal in integrating energy storage systems with power infrastructures. Cell Modules have brought unparalleled scalability, adapting to diverse energy demands. The Communication System has acted as the crucial link, ensuring cohesive operation and real-time responsiveness.
%*******************************************%
\section{Recommendations and Forward Look}\label{sec:recommend}
%*******************************************%
\textbf{Future Work:} Moving forward, design iterations are essential to reinforce the BMS's robustness and establish a fundamental communication system. The encountered UART communication issues underscore the need for a protocol that scales with the increasing number of microcontrollers. Future enhancements should focus on a scalable protocol, improved user interfaces for diagnostics, and sophisticated algorithms for charge balancing and health assessments, to drive the BMS towards technological and user-centric excellence.
%*******************************************%
\section{Denouement}\label{sec:last_CON}
%*******************************************%
Iterative design methodology is key to achieving a communication system capable of supporting the highest number of cell module microcontrollers. Calculating the power ratings for extensive cell networks and designing a controller for series string disconnect/connect are crucial steps. The design, boxed in unit cell string modules, must be fortified at every layer—starting from the main controllers to the high voltage pinnacle. Each iterative layer adds to the robustness of the foundation, making the cell monitoring module integral in a Modular Battery Management System for LFP Batteries, paving the way for an energy-resilient future.
%*******************************************%

\vfill
