\chapter*{Executive Summary\markboth{}{Executive Summary }}
\addcontentsline{toc}{chapter}{Executive Summary}
%*******************************************%
\noindent
The project “A Modular Battery Management System for LFP Batteries" concerns the design, develop, and test a modular Battery Management System (BMS) for Lithium Iron Phosphate (LFP) batteries to optimize cell management. Current BMS solutions face issues like wire overload and inaccurate voltage measurements in high voltage banks due to extended wiring to a single controller. This project presents a BMS design where each cell has a monitoring module with its micro-controller, ensuring precise voltage measurements and module-to-module communication. The improved BMS aims to boost efficiency and accuracy in high voltage banks, benefiting renewable energy, electric vehicles, and energy storage. Potential challenges include logistical, practical constraints, and technical hurdles. However, rigorous testing and the system's adaptability predict success. Future work can expand on this BMS design, adapting it for other batteries or systems. The project's continuation plans comprise thorough documentation, sharing results with academia, and collaboration opportunities for advanced battery management system research.\newline\newline
\noindent
\textbf{Afrikaans:}\newline
\noindent
\emph{Die projek "’n Modulêre Battery Bestuurstelsel vir LFP Batterye" handel oor die ontwerp, ontwikkeling, en toetsing van 'n modulêre Battery Bestuur Stelsel (BBS) vir Litium Yster Fosfaat (LFP) batterye om sel bestuur te optimaliseer. Huidige BBS oplossings ondervind probleme soos draad oorlading en onakkurate spanning metings in hoë spanning banke as gevolg van verlengde bedrading na 'n enkele beheerder. Hierdie projek bied 'n BBS ontwerp waar elke sel 'n moniterings module met sy eie mikroverwerker het, wat akkurate spanning metings en module-tot-module kommunikasie verseker. Die verbeterde BBS beoog om effektiwiteit en akkuraatheid in hoë spanning banke te verhoog, ten voordeel van hernubare energie, elektriese voertuie, en energie stoor. Potensiële uitdagings sluit logistieke, praktiese beperkinge, en tegniese struikelblokke in. Nietemin, deeglike toetsing en die stelsel se aanpasbaarheid voorspel sukses. Toekomstige werk kan voortbou op hierdie BBS ontwerp, deur dit aan te pas vir ander batterye of stelsels. Die projek se voortsettings planne sluit in deeglike dokumentasie, die deel van resultate met die akademie, en samewerkings geleenthede vir gevorderde battery bestuur stelsel navorsing.}


\vfill

